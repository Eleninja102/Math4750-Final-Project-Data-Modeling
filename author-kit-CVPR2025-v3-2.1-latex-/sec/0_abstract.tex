\begin{abstract}
	This project developed a predictive random forest classification model to estimate the likelihood of arrests in Chicago using 2022 crime data. The dataset included over 240,000 records with diverse features such as crime type, location, and timing. The model aimed to improve resource allocation and support decision-making in law enforcement. Three models were created and compared: a baseline model, a balanced class-weight model, and an imbalanced model using SMOTE and undersampling techniques. After hyperparameter tuning and evaluation, the imbalanced model was determined to be the best based on geometric mean and recall scores. Feature importance and SHAP analysis provided insights into the influence of key variables on predictions, enhancing interpretability. This model showcases the potential of machine learning in advancing data-driven crime prevention strategies.
\end{abstract}